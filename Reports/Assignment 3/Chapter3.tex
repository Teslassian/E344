\chapter{Temperature sensor conditioning circuit}\label{sec:temp_sensor}

%**********************************************
\section{Intro} \label{sec:temp_intro}
%**********************************************
The temperature sensor signal presents as DC with \SI{50}{Hz} AC noise superimposed. The sensor output amplitude is too small to act as ADC input and the desired range is only from 34\degree C to 42\degree C. The two aforementioned considerations necessitate amplification of the relevant part of the temperature sensor signal to a \numrange{0}{5} \si{\volt} range. Conditioning circuitry is thus required, namely a filter, an offset removing subcircuit, as well as an amplifier. The filter attenuates the AC noise, the offset subcircuit removes enough of the DC offset as to obtain an output signal centered around \SI{2.5}{\volt}, ensuring the largest possible output swing. Finally, the amplifier increases the magnitude of the input signal in order to be suitable as input for an ADC.

%**********************************************
\section{Design}\label{sec:temp_design}
%**********************************************
The amplifier was designed first, as its gain serves as the determining factor for the amount of noise reduction that is required from the filter. A TLC2272 op-amp was chosen as it allows for an output very close to its rails. Since the output signal has to be centered around \SI{2.5}{\volt}, a differential amplifier was decided upon, allowing for adjustment of the negative input in order to determine the output offset. The zero-reference temperature sensor voltage ($V_{zero}$) is \SI{440}{\milli \volt}, and increases by \SI{35}{\milli \volt} for every 1\degree C ($V_{\Delta}$). Therefore  $V_{temp} = V_{zero} + V_{\Delta} \times Temp$. As shown in table \ref{tab:temp}, the maximum input voltage swing equals $1.91 - 1.63 = 0.28$V, and

\begin{wraptable}{R}{0.65\textwidth}
%\begin{table}[h]
        \centering
%        \vspace{-2cm}
        \footnotesize
        \caption{Temperatures and Corresponding Voltage Levels}
         \begin{tabular}{c@{\qquad}rrr}
          \toprule
          Temperature [\degree C] 	& 32    & 38	& 42\\
          Voltage [V] 			& 1.63	& 1.77	& 1.91\\
          \bottomrule
        \end{tabular}
     \label{tab:temp}
%\end{table}
\end{wraptable}

has a DC offset of \SI{1.77}{\volt}. Amplification is needed to reach an output voltage swing of \SI{5}{\volt}, which, combined with a \SI{2.5}{V} DC offset, ensures that the circuit uses the input range in accordance to specification, while excluding the lower and upper temperature ranges (i.e. below 34\degree C and above 42\degree C). Therefore:

$$A_v = \frac{V_{out}}{V_{in}} = \frac{5}{0.28} = 17.86$$

Considering the current design requirement of \SI{25}{mA} maximum, $R_{input}$ is selected as \SI{10}{\kilo \Omega}, which will therefore, at the highest possible voltage level of \SI{5}{\volt}, draw \SI{0.5}{mA}. This gives $R_{feedback} = 178.6$k$\Omega$, according to $A_v = \frac{{R}_{feedback}}{R}$ \cite{opamp}. $R_{input}$ corresponds to $R_1$ and $R_2$, and $R_{feedback}$ to $R_{3}$ and $R_4$ in the design diagram (figure \ref{fig:final}), which is shown here already as to aid with explanation of the design process. $R_{feedback}$ required adjustment to \SI{230}{k\Omega} upon simulation due to the non-linear $A_v$ curve characteristic of op-amps.

\begin{figure}[H]
    \centering
    \includegraphics[width = 0.8\textwidth]{Figures/final.png}
    \caption{Temperature Sensor Circuit}
    \label{fig:final}
\end{figure}

Two points require consideration: 
\begin{enumerate}
\item The DC offset of the input signal is undesired and has to be removed in order to obtain a zero-mean input signal. This can be achieved by designing another subcircuit that makes use of another op-amp, for example. This adds to the cost and complexity of the circuit.
\item The output signal has to be centered around \SI{2.5}{\volt}. This means that a DC offset has to be added in the form of a virtual ground.
\end{enumerate}

When considered in conjunction with each other, the DC offset alteration can be resolved in one step, thereby reducing cost and complexity significantly. The decision was therefore made to use a differential amplifier with the input signal connected to the positive input, after which the voltage required at the negative input can be calculated in such a way as to simultaneously subtract the offset and add the virtual ground in one step, thereby producing an output DC offset of \SI{2.5}{\volt}.  This approach also simplifies the design procedure, as it renders a separate virtual ground completely superfluous, thereby also removing the need to perform any virtual ground calculations per se.  (Lecture Video 2, minute 11\cite{vground} mentions this to be an acceptable approach). For a non-inverting amplifier, the calculation thus reduces to a simple differential amplifier gain formula \cite{opamp}: 

$$V_{out}=\frac{{R}_{feedback}}{{R}}\left({V}_{in+}-{V}_{in-}\right) \;\;\; \rightarrow \;\;\; 2.5=\frac{230000}{10000}\left(1.77-{V}_{in-}\right)$$

With $V_{out}$ as \SI{2.5}{\volt}, $V_{in+}$ as \SI{1.77}{\volt} and the resistor values as calculated previously, ${V}_{in-} = 1.661 \; \mathrm{V}$. The voltage at ${V}_{in-}$ can be set by means of a voltage divider circuit, which takes \SI{5}{\volt} as input and is calculated as follows (resistor names in formulae are selected to conform with Figure \ref{fig:final}): ${V}_{in-} = 5 (\frac{R_{6}}{R_{6}\times R_{5}})$. Selecting $R_5$ as \SI{10}{\kilo \Omega} gives $R_6 =$ \SI{5.0}{\kilo \Omega}. Here, the common-mode voltage, $V_{IC}$, needs consideration; the TLC2272 can operate with a common-mode voltage of $V_{DD-}$ to $V_{DD+} - 1.5$\cite{tlc2272}. Since $V_{{in+}_{max}}$ is \SI{1.91}{\volt} and $V_{in-}$ is \SI{1.66}{\volt}, the largest possible common mode voltage is $V_{IC} = \frac{1.91 + 1.66}{2} =$ \SI{1.79}{\volt}, which is well below the maximum limit. Since $V_{in-}$ stays constant, the lower limit will never be reached either.\\

Simulation was used to select a filter, and has shown the following: a passive low-pass filter is very simple, but produces too much noise and does not meet the settling time requirement. The active low-pass filter meets both the noise and settling time requirements, but requires the TLC2272 op-amp to do so. Since a single TLC2272 op-amp is more expensive than multiple TL081 op-amps, the decision was made to rather use cascaded second order low-pass filters, using the TL081. This is somewhat more complex, but lowers the cost, as the final circuit now only uses three op-amps, two of which are the cheaper TL081 models. Assignment 3 further motivates the decision for increased filter complexity, as the cascaded setup produces an output signal with extremely low noise, which is indispensable for meeting the bonus requirements during calibration and digitisation - see section \ref{sec:ADCTempEmp}. Following selection, the filter is designed:
\begin{wrapfigure}{r}{0.45\textwidth}
    \centering
    \vspace{-0.5cm}
    \includegraphics[width = 0.44\textwidth]{./Figures/ac}
    \caption{Simulated bode plot of filter}
    \label{fig:ac}
\end{wrapfigure}
A filter gain of close to unity is desired; for $R_A = $ \SI{500}{\kilo\Omega}, $R_A =$ \SI{5}{\kilo\Omega}, according to the formula: ${A_v}=1+\frac{{R}_A}{R_B}$ \cite{filter}. The settling time requirement of \SI{100}{ms} means that a cutoff frequency of more than \SI{10}{Hz} is needed, while the attenuation of noise requires a cutoff frequency below \SI{50}{Hz}. $f_c = $ \SI{15}{Hz} was chosen - the bandwidth thus also is \SI{15}{Hz}. Choosing R ($\mathrm{R_7 \; and \; R_9}$ in the diagram) as \SI{100}{\kilo\Omega} gives C ($\mathrm{C_4 \; and \; C_5}$) as \SI{106.1}{nF}, according to $f_c=\frac{1}{2 \pi RC}$. The given cutoff frequency implies a rise time of \SI{19.1}{ms} according to $t_{r} \approx \frac{1.8}{w_{n}} = \frac{1.8}{2 \pi (15)}$\cite{cs}. This meets the requirement of \SI{100}{ms}, even when cascaded. After design completion, this filter is then duplicated and connected back-to-back in order to form cascaded second-order low-pass filters, as seen in figure \ref{fig:final}. The simulated frequency response is shown in figure \ref{fig:ac}. While \SI{10.5}{Hz} is somewhat lower than designed, it still falls within the range determined by the cutoff frequency and settling time requirements, as discussed previously.\\
Assuming that each resistor is subjected to an average voltage of \SI{2.5}{\volt}, while op-amps draws 3 mA\cite{tlc2272}, the calculated current consumption of the temperature sensor circuit is:

$I_{total} = (3)\frac{2.5}{10k} + (2)\frac{2.5}{230k} + (4)\frac{2.5}{100k} + (2)\frac{2.5}{500k} + (3)\frac{2.5}{5k} + (2)\frac{2.5}{230k} + (3)3mA = 11.38mA$

This leaves \SI{88.62}{mA} for the rest of the health-monitoring system.
 
%**********************************************
\section{Results} \label{sec:temp_results}
%**********************************************
The designed circuit receives an input signal ranging from 1.67 to \SI{1.94}{\volt} for $V_{in+}$, which is somewhat higher than the calculated values of 1.63 to \SI{1.91}{\volt}, and is due to the filter adding some offset. This can easily be overcome by adjusting the voltage divider. $V_{in-}$ receives \SI{1.7}{\volt}. Both $V_{in-}$ and $V_{in+}$ thus fall well within the allowable range for the TLC2272, which is from $\mathrm{V}_{\mathrm{DD}-}-0.3 \;\; \mathrm{to} \;\; \mathrm{V}_{\mathrm{DD}+}$\cite{tlc2272}.  The circuit produces a signal centered at \SI{2.5}{\volt} with an output swing of \SI{4.86}{\volt} (figure \ref{subfig:vout}), exceeding the required \SI{3.5}{\volt}. The absolute maximum amount of noise measured is \SI{25.6}{\milli\volt} (figure \ref{subfig:noise}), and the settling time is \SI{67}{ms} (figure \ref{subfig:ts}), thereby meeting the requirements of \SI{50}{\milli\volt} and \SI{100}{ms} respectively. Total current draw is \SI{12.83}{mA} (figure \ref{subfig:current}), well below the required \SI{15}{mA}, and very close to the calculated \SI{11.38}{mA}. The output signal (light blue) is shown in figure \ref{subfig:vout}.

\begin{figure}[h]
 \footnotesize
 \centering
    \begin{subfigure}[]{0.45\textwidth}
              \centering
  		\includegraphics[width=1\linewidth]{./Figures/vout}
		    \caption{Full Input and Output Range} \label{subfig:vout}
     \end{subfigure}
     \begin{subfigure}[]{0.45\textwidth}
             \centering
  		\includegraphics[width=1\linewidth]{./Figures/current1}
		   \caption{Current of Entire System} \label{subfig:current}
     \end{subfigure}
    \begin{subfigure}[]{0.45\textwidth}
              \centering
              \vspace{0.5cm}
  		\includegraphics[width=1\linewidth]{./Figures/noise}
		    \caption{Simulated Noise Suppresion} \label{subfig:noise}
     \end{subfigure}
    \begin{subfigure}[]{0.45\textwidth}
              \centering
              \vspace{0.5cm}
  		\includegraphics[width=1\linewidth]{./Figures/ts}
		    \caption{Simulated Output Rise Time} \label{subfig:ts}
     \end{subfigure}
   \caption[\textcolor{red}{Complete Circuit Output}]{Complete Circuit Output (a) Output Range: 0.1 to 4.85V. Input Range: 1.63 to 1.94 V. (b)  Current draw below requirement.  (c)  Noise suppressed to negligible levels. (d) Settling time within 100 ms.}
    \label{fig:simulation_results_box}
 \end{figure}

%**********************************************
\section{Summary}\label{sec:temp_summary}
%**********************************************
Concluding, the circuit performs very well and successfully amplifies the temperature sensor output to a level that is readable by the microcontroller ADC, all the while attenuating almost all noise present in the input. The design is somewhat more complex, which however is justified by the requirements of Assignment 3. While being complex, it still is inexpensive, as it uses less of the TLC2272 op-amps. The design meets all base and bonus requirements requirements with a good margin to spare, all the while using only three op-amps, two of which are the cheaper TL081 models.


